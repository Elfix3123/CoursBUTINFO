\documentclass[12pt, a4paper, french]{article}
\usepackage[utf8]{inputenc}
\usepackage[T1]{fontenc}
\usepackage{babel}
\usepackage[pdftex]{graphicx, color}
\usepackage[shortlabels]{enumitem}

\title{SAE1.04 \item Naufrage du Titanic \item Etape 1}
\author{\bsc{F.Martins} \and \bsc{N.Rayot}}
\date{\today}

\begin{document}

\selectlanguage{french}

\begin{titlepage}
	\begin{center}
		\textbf{SAE1.04 \item Naufrage du Titanic \item Etape 2}
		\vspace*{1cm}
		
		\today
		\vspace*{1cm}
		
		IUT2 \item Dép. Informatique
		\vfill
		\raggedleft

		F.Martins, N.Rayot
	\end{center}
 \end{titlepage}
\thispagestyle{empty}
\newpage

\tableofcontents
\setcounter{page}{2}
\thispagestyle{empty}
\begin{sloppypar}
\newpage

\section{Justification du SLR}

\begin{itemize}
\item Dans PORT:
\begin{itemize}
    \item PortId, PortName et Country viennent de l'application de la règle R0 à l'entité PORT. 
	\item PortId est clef primaire car il représente l'identifiant de PORT.
\end{itemize}
\item Dans PASSENGER:
\begin{itemize}
    \item PassengerId, Name, Sex, Age, Survived viennent de l'application de la règle R0 à l'entité PASSENGER. 
	\item PassengerId est clef primaire car il représente l'identifiant de PASSENGER.
    \item PortId provient de l'application de la regle R1 à l'association BOARDING qui associe seulement une occurrence de l'entité PORT à une occurrence de l'entité PASSENGER.
    \item PortId est une clef étrangère en tant qu'identifiant de PORT.
    \item PClass n'est pas une clef étrangère car il appartenait à une entité contenant un seul élément et était en clef étrangère seulement dans l'entité PASSENGER.
\end{itemize}
\item Dans OCCUPATION:
\begin{itemize}
    \item PassengerId et CabinCode deviennent clef primaire d'OCCUPATION avec l'application de la R3.
    \item PassengerId est une clef étrangère en tant qu'identifiant de PASSENGER avec l'application de la R3.
    \item CabinCode n'est pas une clef étrangère car il appartenait à une entité contenant un seul élément et était en clef étrangère seulement dans l'entité OCCUPATION.
\end{itemize}
\item Dans SERVICE:
\begin{itemize}
    \item Role résulte de l'application de la R0 car Role était un attribut de l'association SERVICE.
    \item PassengerId\_Dom, PassengerId\_Emp sont des clef étrangères qui proviennent de PASSENGER suite à la R3. 
	\item PassengerId\_Dom est clef primaire par l'application des cardinalités.
\end{itemize}
\item Dans CATEGORY:
\begin{itemize}
    \item LifeBoatCat,Structure et Places viennent de l'application de la règle R0 à l'entité CATEGORY.
    \item LifeBoatCat est la clef primaire car identifiant de l'entité CATEGORY.
\end{itemize}
\item Dans LIFEBOAT:
\begin{itemize}
    \item LifeBoatId ,Side, Position et Location viennent de l'entité LIFEBOAT suite a l'application de R0.
	\item LifeBoatId est clef primaire en tant qu'identifiant de LIFEBOAT.
    \item Lauching\_Time  vient de l'association OBSERVED\_TIME suite a l'applicationd de la R3. 
	\item Elle n'est pas en clef étrangère car la relation OBSERVED\_TIME est supprimée puisqu'elle n'a qu'un attribut.
    \item LifeBoatCat est une clef étrangère de CATEGORY. Elle est présente dans la relation suite à l'application de la R3.
\end{itemize}
\item Dans RECOVERY:
\begin{itemize}
	\item Recovery\_Time vient de l'association OBSERVED\_TIME suite à l'application de la R3. Elle n'est pas en clef étrangère suite à la supression de OBSERVED\_TIME.
    \item LifeBoatId est une clef étrangère de l'entité LIFEBOAT avec l'application de la R2.
\end{itemize}
\item Dans RESCUE:
\begin{itemize}
    \item PassengerId est une clef étrangère en tant qu'identifiant de PASSENGER suite à l'application de la R2.
    \item LifeBoatId est une clef étrangère en tant qu'identifiant de LIFEBOAT suite à l'application de R3.
\end{itemize}

\end{itemize}
\newpage
\section{Contraintes du SLR}

Sur consigne, on ignorera les contraintes "NOT NULL" et les contraintes de type.

\begin{itemize}

\item  Port :
\begin{itemize}
\item Contrainte d'attribut sur PortId : PRIMARY KEY(PortId)
\item Contrainte d'attribut sur PortId : CHECK(PortId IN ('C', 'Q', 'S'))
\end{itemize}
\item Passenger :
\begin{itemize} 
\item Contrainte d'attribut sur PassengerId : PRIMARY KEY(PassengerId)
\item Contrainte d'attribut sur Survived : CHECK(Survived in (0, 1))
\item Contrainte d'attribut sur PClass : CHECK(1 <= PClass AND PClass <= 3)
\item Contrainte référentielle sur PortId : PortId REFERENCES PORT(PortId)
\end{itemize}
\item Occupation :
\begin{itemize}
\item Contrainte référentielle sur PassengerId : PassengerId REFERENCES PASSENGER(PassengerId)
\item Contrainte de relation sur PassengerID et CabinCode : PRIMARY KEY (PassengerId, CabinCode)
\end{itemize}
\item Service :
\begin{itemize}
\item Contrainte d'attribut sur PassengerId\_Dom : PRIMARY KEY(PassengerId\_Dom) lalalalalalalalal
\item Contrainte référentielle sur PassengerId\_Dom : PassengerId\_Dom REFERENCES PASSENGER(PassengerId)
\item Contrainte référentielle sur PassengerId\_Emp : PassengerId\_Emp REFERENCES PASSENGER(PassengerId)
\end{itemize}
\item Category :
\begin{itemize}
\item Contrainte d'attribut sur LifeBoatCat : PRIMARY KEY(LifeBoatCat)
\item Contrainte d'attribut sur LifeBoatCat : CHECK(LifeBoatCat in ('standard', 'secours', 'radeau'))
\item Contrainte d'attribut sur Structure : CHECK(Structure in ('bois', 'bois et toile'))
\end{itemize}
\item Lifeboat :*
\begin{itemize}
\item Contrainte d'attribut sur LifeBoatId : PRIMARY KEY(LifeBoatId)
\item Contrainte référentielle sur LifeBoatCat : LifeBoatCat REFERENCES CATEGORY(LifeBoatCat)
\item Contrainte d'attribut sur Side : CHECK(Side in ('babord', 'tribord'))
\item Contrainte d'attribut sur Position : CHECK(Position in ('avant', 'arriere'))
\end{itemize}
\item Recovery :
\begin{itemize}
\item Contrainte d'attribut sur LifeBoatId : PRIMARY KEY(LifeBoatId)
\item Contrainte référentielle sur LifeBoatId : LifeBoatId REFERENCES LIFEBOAT(LifeBoatId)
\end{itemize}
\item Rescue :
\begin{itemize}
\item Contrainte d'attribut sur PassengerId : PRIMARY KEY(PassengerId)
\item Contrainte référentielle sur PassengerId : PassengerId REFERENCES PASSENGER(PassengerId)
\item Contrainte référentielle sur LifeBoatId : LifeBoatId REFERENCES LIFEBOAT(LifeBoatId)
\end{itemize}

\end{itemize}
\end{sloppypar}
\end{document}