% Options for packages loaded elsewhere
\PassOptionsToPackage{unicode}{hyperref}
\PassOptionsToPackage{hyphens}{url}
%
\documentclass[
]{article}
\usepackage{amsmath,amssymb}
\usepackage{iftex}
\ifPDFTeX
  \usepackage[T1]{fontenc}
  \usepackage[utf8]{inputenc}
  \usepackage{textcomp} % provide euro and other symbols
\else % if luatex or xetex
  \usepackage{unicode-math} % this also loads fontspec
  \defaultfontfeatures{Scale=MatchLowercase}
  \defaultfontfeatures[\rmfamily]{Ligatures=TeX,Scale=1}
\fi
\usepackage{lmodern}
\ifPDFTeX\else
  % xetex/luatex font selection
\fi
% Use upquote if available, for straight quotes in verbatim environments
\IfFileExists{upquote.sty}{\usepackage{upquote}}{}
\IfFileExists{microtype.sty}{% use microtype if available
  \usepackage[]{microtype}
  \UseMicrotypeSet[protrusion]{basicmath} % disable protrusion for tt fonts
}{}
\makeatletter
\@ifundefined{KOMAClassName}{% if non-KOMA class
  \IfFileExists{parskip.sty}{%
    \usepackage{parskip}
  }{% else
    \setlength{\parindent}{0pt}
    \setlength{\parskip}{6pt plus 2pt minus 1pt}}
}{% if KOMA class
  \KOMAoptions{parskip=half}}
\makeatother
\usepackage{xcolor}
\usepackage[margin=1in]{geometry}
\usepackage{graphicx}
\makeatletter
\def\maxwidth{\ifdim\Gin@nat@width>\linewidth\linewidth\else\Gin@nat@width\fi}
\def\maxheight{\ifdim\Gin@nat@height>\textheight\textheight\else\Gin@nat@height\fi}
\makeatother
% Scale images if necessary, so that they will not overflow the page
% margins by default, and it is still possible to overwrite the defaults
% using explicit options in \includegraphics[width, height, ...]{}
\setkeys{Gin}{width=\maxwidth,height=\maxheight,keepaspectratio}
% Set default figure placement to htbp
\makeatletter
\def\fps@figure{htbp}
\makeatother
\setlength{\emergencystretch}{3em} % prevent overfull lines
\providecommand{\tightlist}{%
  \setlength{\itemsep}{0pt}\setlength{\parskip}{0pt}}
\setcounter{secnumdepth}{-\maxdimen} % remove section numbering
\ifLuaTeX
  \usepackage{selnolig}  % disable illegal ligatures
\fi
\usepackage{bookmark}
\IfFileExists{xurl.sty}{\usepackage{xurl}}{} % add URL line breaks if available
\urlstyle{same}
\hypersetup{
  pdftitle={Maximisation d'une fonction linéaire},
  hidelinks,
  pdfcreator={LaTeX via pandoc}}

\title{Maximisation d'une fonction linéaire}
\author{}
\date{\vspace{-2.5em}17/05/2024}

\begin{document}
\maketitle

\section{Introduction}\label{introduction}

Dans le cadre de notre organisation de tournoi, nous cherchons à vendre
du merch pendant les évènements physiques. Nous décidons dans ce cadre
de vendre deux types de merch, des tshirt et des magnets, et posons les
prix de ceux-ci à 5 euros pour un magnet et 12 euros pour un tshirt.
Nous cherchons à maximiser l'argent récolté dans le cas où chacunes des
unités préparées est vendue. Dans ce cadre, nous posons la fonction
\(f(x, y) = 5x + 12y\) qui représente l'argent récolté et où \(x\)
représente le nombre de magnets vendus et \(y\) le nombre de tshirts
vendus. La question de la maximisation de l'argent récolté devient alors
celle de la maximisation de la fonction \(f\).

\section{Contraintes}\label{contraintes}

Nous posons cependant certaines contraintes qui limitent notre champs
d'action :

\begin{itemize}
\item
  Le nombre d'unités de chaque type doit être positif.
\item
  En raison de la taille du camion, le nombre total d'unités doit être
  inférieur à 50.
\item
  En raison du poids maximal transportable par nos équipements, le poids
  total des unités doit être inférieur à 15kg, sachant que chaque magnet
  pèse 100g et que chaque tshirt pèse 500g.
\end{itemize}

Nous formalisons les contraintes précédentes comme suit : \[
\begin{cases}
x,y\geq 0\\
x + y \leq 50 \\
0.1x + 0.5y \leq 15
\end{cases}
\]

\section{Résolution graphique}\label{ruxe9solution-graphique}

Afin de résoudre graphiquement le maximum de \(f\) qui réspecte les
contraintes énnoncées précédemment, on représente le domaine des
contraintes dans le plan :

\includegraphics{SAE1256_09_files/figure-latex/unnamed-chunk-1-1.pdf}

Nous avons ici représenté \(y = -x + 5\) en bleu, \(y = 0.2x + 30\) en
rouge, \(x = 0\) en orange et \(y = 0\) en vert. Avec ces
représentations graphiques, nous pouvons visuellement voir nos
contraintes; ainsi, nous pouvons représenter graphiquement la fonction,
qu'on déplace afin de trouver les valeurs de \(x\) et \(y\) qui la
maximisent.

\includegraphics{SAE1256_09_files/figure-latex/unnamed-chunk-2-1.pdf}

Nous avons ici représenté \(y = \frac{-5x}{12}\) en violet et
\(y = \frac{-5x}{12} + 35.5\) en rose. La droite violette représente
ainsi la fonction qu'on cherche à maximiser et la droite rose une
fonction qui en découle, modifié afin qu'un point de celle-ci soit le
maximum qui correspond à nos contraintes. Ainsi, nous pouvons déduire
graphiquement que les valeurs de \(x\) et \(y\) qui maximisent la
fonction \(f(x, y) = 5x + 12y\) tout en respectant nos contraintes sont
\(x = 25\) et \(y = 25\).

\section{Résolution par simplex}\label{ruxe9solution-par-simplex}

Nous cherchons désoirmais à résoudre le problème de maximisation de
\(f\) en utilisant la méthode du simplex.

\begin{verbatim}
## 
## Linear Programming Results
## 
## Call : simplex(a = z, A1 = A1, b1 = b1, maxi = TRUE)
## 
## Maximization Problem with Objective Function Coefficients
## x1 x2 
##  5 12 
## 
## 
## Optimal solution has the following values
## x1 x2 
## 25 25 
## The optimal value of the objective  function is 425.
\end{verbatim}

\section{Conclusion}\label{conclusion}

\end{document}
